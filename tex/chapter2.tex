\documentclass[a4j,11pt]{jreport}
\usepackage{graphicx}
%\usepackage{wrapfig}
\usepackage{amsmath,amsthm,amssymb}
\usepackage{pifont}
\usepackage{bm}
\usepackage{ascmac}
\usepackage{color}
\usepackage[hypertex]{hyperref}
%\usepackage{hyperref}

\begin{document}

\chapter{hoge}

\chapter{Transverse Ising Chain}

\section{Symmetries and the Critical Point}

\subsection{Duality Symmetry of the Transverse Ising Model}

The duality transformation.
\begin{eqnarray}
\tau_{j}^x = S_{j}^z S_{j+1}^z \\
\tau_{j}^z = \prod_{k \le j} S_k^x \ .
\end{eqnarray}

(反)交換関係を確認する。

\begin{eqnarray}
 \{ \tau_{i}^x , \tau_{i}^z  \} &=& S_{i}^z S_{i+1}^z \left ( \prod_{k \le i} S_{k}^x \right ) + \left ( \prod_{k \le i}  S_{k}^x \right ) S_{i}^z S_{i+1}^z \nonumber \\
&=&
S_{i}^z S_{i}^x S_{i+1} \left ( \prod_{k < i} S_{k}^x \right ) + \left ( \prod_{k < i}  S_{k}^x \right ) S_{i}^x S_{i}^z S_{i+1}^z \nonumber \\
&=&
- S_{i}^x S_{i}^z  S_{i+1} \left ( \prod_{k < i} S_{k}^x \right ) + \left ( \prod_{k < i}  S_{k}^x \right ) S_{i}^x S_{i}^z S_{i+1}^z = 0 \ .
\end{eqnarray}
3番目の等号は $S_i^x, S_i^z$ についての反交換関係を使った。

交換関係については、$\tau_{i}^x, \tau_{j}^z$ 内それぞれにに同じ site の spin が含まれなければ明らかに交換する。$i \neq j$であり、かつ、同じ site の spin が含まてている場合 $ i < j $ を考える。

\begin{eqnarray}
 \left [
\tau_{i}^x, \tau_{j}^z
\right ]
&=&
\left [
S_{i}^z S_{i+1}^z, \prod_{k < j} S_k^x
 \right ] \nonumber \\
&=&
S_{1}^x \cdots S_{i-1}^x
 \left [
S_{i}^z S_{i+1}^z,
S_{i}^x S_{i+1}^x
 \right ]
S_{i+2}^x \cdots S_{j-1}^x \ .
\end{eqnarray}

この中で注意しなければならないのは、$k=i$ となる場合の連続する $k, k+1$ 番目についてである。

\begin{eqnarray}
 \left [
S_{i}^z S_{i+1}^z,
S_{i}^x S_{i+1}^x
 \right ] &=&
S_{i}^z S_{i+1}^z S_{i}^x S_{i+1}^x - S_{i}^x S_{i+1}^x S_{i}^z S_{i+1}^z \nonumber \\
&=&
S_{i}^z S_{i}^x  S_{i+1}^z S_{i+1}^x - S_{i}^x S_{i}^z S_{i+1}^x  S_{i+1}^z \nonumber \\
&=&
S_{i}^z S_{i}^x  S_{i+1}^z S_{i+1}^x - S_{i}^z S_{i}^x  S_{i+1}^z S_{i+1}^x \nonumber \\
&=& 0 \ .
\end{eqnarray}

\end{document}
